\section{Электро(статика/динамика), теория поля}
	\subsection{Электростатика}
		\subsubsection{Теорема Гаусса}
		
		Поток вектора $\overrightarrow{E}$ через произвольную замкнутую поверхность зависит только от суммы зарядов, находящихся внутри этой поверхности.
		
\begin{equation} \label{eq:int_Gauss}
\oint_{S} \overrightarrow{E} d \overrightarrow{S} = \frac{q_{in}}{\epsilon_{0}}
\end{equation}

Разделим правую и левую часть уравнения (\ref{eq:int_Gauss}) на объем $V$ и устремим его к 0.

\begin{equation}
\lim_{V -> 0} \frac{1}{V} \oint_{S} \overrightarrow{E} d \overrightarrow{S} = \frac{<p>}{\epsilon_0}
\end{equation}

Согласно определению, отношение потока через замкнутую поверхность к объему этой поверхности при стремлении объёма к нулю есть дивергенция вектора поля.

\begin{equation}
div \overrightarrow{E} =  \lim_{V -> 0} \frac{1}{V} \oint_{S} \overrightarrow{E} d \overrightarrow{S}
\end{equation}

Таким образом:

\begin{equation}
div \overrightarrow{E} = \frac{p}{\epsilon_0}
\end{equation}

или

\begin{equation} \label{eq:dif_Gauss}
\nabla \cdot  \overrightarrow{E} = \frac{p}{\epsilon_0}
\end{equation}

, где $p$ - плотность заряда в точке

$\nabla =  \overrightarrow{i} \frac{\partial}{\partial x} + \overrightarrow{j} \frac{\partial}{\partial y} + \overrightarrow{k} \frac{\partial}{\partial z}$

Уравнение (\ref{eq:dif_Gauss}) есть запись уравнения Гаусса в дифференциальной форме.


\subsubsection{Потенциал}


Из механики известно, что любое поле центральных сил я является консервативным, т.е. работа силы при перемещении по замкнутому контуру равняется нулю.

Из этого следует теорема о циркуляции:

Циркуляция вектора $\overrightarrow{E}$ по любому замкнутому контуру равняется нулю.

\begin{equation}
\oint \overrightarrow{E} d \overline{l} = 0 
\end{equation}

Поле, обладающее данным свойством, называют потенциальным.
Любое электростатическое поле является потенциальным.

Потенциал - это величина, численно равная энергии частицы с единичным положительным зарядом в данной точке поля.

\begin{equation}
\int_1^2 \overrightarrow{E} d \overline{l} = \phi_1 - \phi_2
\end{equation}

Пусть проекция приращение расстояние $dl$ на ось X есть $\overline{i} dx$. Тогда
\[E_xW = - \frac{\partial \phi}{\partial x}\] 
отсюда:

\begin{equation}
\overline{E} =  - \nabla \phi
\end{equation}

\subsubsection{Диполь}

Диполь - система одинаковых по модулю, разноименных точечных зарядов, находящихся на некотором расстоянии l друг от друга.
Предполагается, что расстояние между зарядами много меньше, чем до некоторых, интересующих нас точек.

Поле диполя обладает осевой симметрией.

Потенциал поля:


\begin{equation}
\phi = \frac{q l \cos(\gamma)}{4 \pi \epsilon_o r^2}
\end{equation}

Где $\gamma$ - угол между осью диполя и прямой, соединяющей точки и геометрический центр диполя.

\[p = ql\]
 - электрический момент диполя.

Модуль напряженности поля, создаваемого диполем можно определить по формуле:

\begin{equation}
E = \sqrt{E_r^2 + E_\gamma^2}
\end{equation}

Где
\[E_r = - \frac{\partial \phi}{\partial r}\]
\[E_\gamma = - \frac{\partial \phi}{r \partial \gamma}\]

Сила, действующая на заряд диполя равняется разности напряженности поля в точке положительного и отрицательного зарядов.
Таким образом, так как расстояние между зарядами мало, то

\begin{equation}
\overline{F} = q l \frac{\partial \overline{E}}{\partial l} = p \frac{\partial \overline{E}}{\partial l}
\end{equation}
Где $\frac{\partial \overline{E}}{\partial l}$ - производная по направлению l.

Момент сил, действующих на диполь:


\begin{equation}
M = [\overline{p},\overline{E}]
\end{equation}

Энергия диполя:

\begin{equation}
W = -\overline{p} \overline{E}
\end{equation}

