\section{Электро(статика/динамика), теория поля}
	\subsection{Электростатика}
		\subsubsection{Теорема Гаусса}
		
		Поток вектора $\overrightarrow{E}$ через произвольную замкнутую поверхность зависит только от суммы зарядов, находящихся внутри этой поверхности.
		
\begin{equation} \label{eq:int_Gauss}
\oint_{S} \overrightarrow{E} d \overrightarrow{S} = \frac{q_{in}}{\epsilon_{0}}
\end{equation}

Разделим правую и левую часть уравнения (\ref{eq:int_Gauss}) на объем $V$ и устремим его к 0.

\begin{equation}
\lim_{V -> 0} \frac{1}{V} \oint_{S} \overrightarrow{E} d \overrightarrow{S} = \frac{<p>}{\epsilon_0}
\end{equation}

Согласно определению, отношение потока через замкнутую поверхность к объему этой поверхности при стремлении объёма к нулю есть дивергенция вектора поля.

\begin{equation}
div \overrightarrow{E} =  \lim_{V -> 0} \frac{1}{V} \oint_{S} \overrightarrow{E} d \overrightarrow{S}
\end{equation}

Таким образом:

\begin{equation}
div \overrightarrow{E} = \frac{p}{\epsilon_0}
\end{equation}

или

\begin{equation} \label{eq:dif_Gauss}
\nabla \cdot  \overrightarrow{E} = \frac{p}{\epsilon_0}
\end{equation}

, где $p$ - плотность заряда в точке

$\nabla =  \overrightarrow{i} \frac{\partial}{\partial x} + \overrightarrow{j} \frac{\partial}{\partial y} + \overrightarrow{k} \frac{\partial}{\partial z}$

Уравнение (\ref{eq:dif_Gauss}) есть запись уравнения Гаусса в дифференциальной форме.


\subsubsection{Потенциал}

