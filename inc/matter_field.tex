\subsection{Поле в веществе}

\subsubsection{Поле в проводнике}


Т.к. в проводнике присутствуют свободные заряды, то при наличии в проводнике электрического поля, заряды выстраиваются таким образом, чтобы компенсировать данное поле.
Следовательно, электрическое поле в проводнике отсутствует.

Поле у поверхности проводника может быть определено с помощью теоремы Гаусса. Рассмотрим цилиндр, перпендикулярный поверхности и пересекающий ее. Тогда поток $\overline{E}$ через поверхность цилиндра будет определяться только потоком через его основание, не лежащее в проводнике. Т.о. 

\begin{equation}
\Delta S E = \frac{\Delta S \sigma}{\epsilon_o}
\end{equation}

\begin{equation}
E_n = \frac{\sigma}{\epsilon_0}
\end{equation}


\subsubsection{Конденсаторы, электроемкость}


Электроемкостью уединенного проводника называют отношение заряда проводника к потенциалу, создаваемому проводником. При условии, что потенциал на бесконечности равен нулю.

\begin{equation}
C=\frac{q}{\phi}
\end{equation}

Конденсатор.

Конденсатор - система тел, емкость которых значительно больше, чем емкость уединенных проводников.

Простейший конденсатор состоит из двух металлических пластин, находящихся на близком расстоянии друг от друга.

Емкость конденсатора понимается как отношение заряда положительно обкладки к разности потенциала между обкладками.

Например, емкость плоского конденсатора
\[С = \frac{\epsilon_0 S}{h}\]

\subsubsection{поле в диэлектрике}

Поле в диэлектрике является суперпозицией сторонних и связанных зарядов.

Поляризованность диэлектрика есть усредненный по объему электрический момент диэлектрика. Т.е. сумма моментов диполей в данном объеме, отнесенная к этому объему.

Для обширного класса диэлектриков, поляризованность линейно зависит от напряженности:
\begin{equation}
\overline{P} = \chi \epsilon_0 \overline{E}
\end{equation}
где $\chi$ - диэлектрическая восприимчивость

Теорема Гаусса для вектора $\overline{P}$.

\begin{equation}
\oint_s \overline{P} d \overline{S} = -q'_{inner}
\end{equation}

Поток вектора поляризованности через замкнутую поверхность равен связанному избыточному заряду диэлектрика, взятого с обратным знаком.

В дифференциальной форме:

\begin{equation}
\nabla \cdot \overline{P} = -p'
\end{equation}




Далее, только формулы, ибо не успеваю.


\begin{equation}
\oint_s (\epsilon_0 \overline{E}) = (q + q')_{inner}
\end{equation}


\begin{equation}
\oint_s (\epsilon_0 \overline{E} + \overline{P}) = q_{inner}
\end{equation}


\begin{equation}
\overline{D} = \epsilon_0 \overline{E} + \overline{P}
\end{equation}


\begin{equation}
\nabla \cdot \overline{D} = p
\end{equation}


\begin{equation}
\overline{D} = \epsilon_0 (1+\chi) \overline{E}
\end{equation}

\begin{equation}
\overline{D} = \epsilon \epsilon_0 \overline{E}
\end{equation}

Граничные условия для векторов E и D определяются из рассмотрения теоремы о циркуляции и теоремы Гаусса для данных векторов соответственно.


\begin{equation}
W = \frac{1}{2} \sum q_i \phi_i 
\end{equation}
Где $\phi$ - потенциал в точке и-того заряда, создаваемого всеми остальными зарядами

\begin{equation}
W = \frac{q\phi}{2}
\end{equation}

\begin{equation}
W = \frac{qU}{2} = \frac{C U^2}{2}
\end{equation}


\begin{equation}
w = \frac{\epsilon \epsilon_0 E^2}{2} = \frac{\overline{ED}}{2}
\end{equation}
