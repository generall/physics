\subsection{Ток, магнитное поле}
Принцип непрерывности.

\begin{equation}
\oint_S \overline{j} d \overline{S} = -\frac{dq}{dt}
\end{equation}

Сила Лоренца

\begin{equation}
\overline{F} = q \overline{E} + q [\overline{v B}]
\end{equation}

\begin{equation}
\overline{B}  = \frac{\mu_0}{4  \pi} \frac{q [\overline{v, r}]}{r^3}
\end{equation}


Закон Био-Савара

\begin{equation}
d\overline{B} =  \frac{\mu_0}{4  \pi}  \frac{I [\overline{dl, r}]}{r^3}
\end{equation}

Поле бесконечного проводника:
\begin{equation}
\overline{B} =  \frac{\mu_0}{4  \pi} \frac{2I}{b}
\end{equation}


Теорема Гаусса для B

\begin{equation}
\oint_s \overline{B} d \overline{S} = 0
\end{equation}

Теорема о циркуляции

\begin{equation}
\oint_l \overline{B} d \overline{l} = \mu_0 I
\end{equation}

\begin{equation}
\nabla \times \overrightarrow{B} =  \mu_0 \overrightarrow{I}
\end{equation}

Закон ампера

\begin{equation}
d \overrightarrow{F} = I [d \overrightarrow{l},\overrightarrow{B} ]
\end{equation}


\begin{equation}
\oint \overrightarrow{J} d \overrightarrow{l} = I'
\end{equation}

\begin{equation}
\oint \overrightarrow{B} d \overrightarrow{l} = \mu_0 (I+I')
\end{equation}

\begin{equation}
\oint ( \frac{\overrightarrow{B}}{\mu_0} - \overrightarrow{J}) d \overrightarrow{l} = I
\end{equation}

\begin{equation}
\overrightarrow{H} = \frac{\overrightarrow{B}}{\mu_0} - \overrightarrow{J}
\end{equation}


\begin{equation}
\oint \overrightarrow{H} d \overrightarrow{l} = I
\end{equation}


\begin{equation}
\overrightarrow{B} = \mu \mu_0 \overrightarrow{H}
\end{equation}


\subsection{Индукция}


