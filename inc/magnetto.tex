\subsection{Ток, магнитное поле}
Принцип непрерывности.

\begin{equation}
\oint_S \overline{j} d \overline{S} = -\frac{dq}{dt}
\end{equation}

Сила Лоренца

\begin{equation}
\overline{F} = q \overline{E} + q [\overline{v B}]
\end{equation}

\begin{equation}
\overline{B}  = \frac{\mu_0}{4  \pi} \frac{q [\overline{v, r}]}{r^3}
\end{equation}


Закон Био-Савара

\begin{equation}
d\overline{B} =  \frac{\mu_0}{4  \pi}  \frac{I [\overline{dl, r}]}{r^3}
\end{equation}

Поле бесконечного проводника:
\begin{equation}
\overline{B} =  \frac{\mu_0}{4  \pi} \frac{2I}{b}
\end{equation}


Теорема Гаусса для B

\begin{equation}
\oint_s \overline{B} d \overline{S} = 0
\end{equation}

Теорема о циркуляции

\begin{equation}
\oint_l \overline{B} d \overline{l} = \mu_0 I
\end{equation}

\begin{equation}
\nabla \times \overrightarrow{B} =  \mu_0 \overrightarrow{I}
\end{equation}

Закон ампера

\begin{equation}
d \overrightarrow{F} = I [d \overrightarrow{l},\overrightarrow{B} ]
\end{equation}


\begin{equation}
\oint \overrightarrow{J} d \overrightarrow{l} = I'
\end{equation}

\begin{equation}
\oint \overrightarrow{B} d \overrightarrow{l} = \mu_0 (I+I')
\end{equation}

\begin{equation}
\oint ( \frac{\overrightarrow{B}}{\mu_0} - \overrightarrow{J}) d \overrightarrow{l} = I
\end{equation}

\begin{equation}
\overrightarrow{H} = \frac{\overrightarrow{B}}{\mu_0} - \overrightarrow{J}
\end{equation}


\begin{equation}
\oint \overrightarrow{H} d \overrightarrow{l} = I
\end{equation}


\begin{equation}
\overrightarrow{B} = \mu \mu_0 \overrightarrow{H}
\end{equation}


\subsection{Индукция}
\subsection{Максвелл}


\begin{equation}
\overrightarrow{j_{full}} = \overrightarrow{j} + \frac{\partial \overrightarrow{D}}{\partial t}
\end{equation}

\begin{equation}
\oint \overrightarrow{H} d \overrightarrow{l} = \int (\overrightarrow{j} + \frac{\partial \overrightarrow{D}}{\partial t}) d\overrightarrow{S}
\end{equation}


\begin{equation}
\nabla \times \overrightarrow{H} = \overrightarrow{j} + \frac{\partial \overrightarrow{D}}{\partial t}
\end{equation}


Уравнения Максвелла.


\begin{equation}
\oint \overrightarrow{E} d \overrightarrow{l} = - \oint  \frac{\partial \overrightarrow{B}}{\partial t} d\overrightarrow{S}
\end{equation}

\begin{equation}
\oint \overrightarrow{H} d \overrightarrow{l} = \int (\overrightarrow{j} + \frac{\partial \overrightarrow{D}}{\partial t}) d\overrightarrow{S}
\end{equation}


\begin{equation}
\oint \overrightarrow{D} d \overrightarrow{S} = \int p d V
\end{equation}


\begin{equation}
\oint \overrightarrow{B} d \overrightarrow{S} = 0
\end{equation}


Уравнения Максвелла в дифференциальной форме.


\begin{equation}
\nabla \times \overrightarrow{E} = - \frac{\partial \overrightarrow{B}}{\partial t}
\end{equation}

\begin{equation}
\nabla \times \overrightarrow{H} = \overrightarrow{j} + \frac{\partial \overrightarrow{D}}{\partial t}
\end{equation}

\begin{equation}
\nabla \cdot \overrightarrow{D} = p
\end{equation}


\begin{equation}
\nabla \cdot \overrightarrow{B} = 0
\end{equation}

Материальные уравнения


\begin{equation}
\overrightarrow{D} = \varepsilon \varepsilon_0 \overrightarrow{E}
\end{equation}

\begin{equation}
\overrightarrow{B} = \mu \mu_0 \overrightarrow{H}
\end{equation}

\begin{equation}
\overrightarrow{j} = \sigma (\overrightarrow{E} + \overrightarrow{E}^*)
\end{equation}


Теорема Пойнтинга:


\begin{equation}
- \frac{d W}{d t} = \oint \overrightarrow{S} d \overrightarrow{A} + P
\end{equation}

Где S - плотность потока энергии А - элемент поверхности P - работа поля над зарядом


\begin{equation}
\overrightarrow{S} = [\overrightarrow{E} \overrightarrow{H} ]
\end{equation}


Импульс появляется вследствие воздействия магнитного поля волны на ток в проводнике, индуцируемый электрическим полем волны.

\begin{equation}
\overrightarrow{j} = \sigma \overrightarrow{E}
\end{equation}


\begin{equation}
\overrightarrow{F} = [\overrightarrow{j} \overrightarrow{B}] = \sigma [\overrightarrow{E} \overrightarrow{B}]
\end{equation}

Плотность импульса

\begin{equation}
\overrightarrow{G} = \frac{\overrightarrow{S}}{c^2}
\end{equation}

\subsubsection{Волновое уравнение}

Выводится из уравнений Максвелла в дифференциальной форме.


\begin{eqnarray}
\nabla^2 \overrightarrow{E} = \varepsilon \varepsilon_0 \mu \mu_0  \frac{\partial^2 \overrightarrow{E}}{ \partial t^2} \\
\nabla^2 \overrightarrow{H} = \varepsilon \varepsilon_0 \mu \mu_0 \frac{\partial^2 \overrightarrow{H}}{ \partial t^2}
\end{eqnarray}


Для плоской волны выполняется условие


\begin{eqnarray}
\frac{\partial \overrightarrow{E_y} } {\partial x} = - \mu \mu_0 \frac{\partial \overrightarrow{H_z}} {\partial t} \\
\frac{\partial  \overrightarrow{H_z}} {\partial x} = - \varepsilon \varepsilon_0 \frac{\partial  \overrightarrow{E_y}} {\partial t}
\end{eqnarray}

Если обозначить 

\begin{equation}
E_y = E_y (t - x/v)
\end{equation}
\begin{equation}
H_z = H_z (t - x/v)
\end{equation}

\begin{equation}
\sqrt{\varepsilon \varepsilon_0} E_y = \sqrt{\mu \mu_0 } H_z
\end{equation}
